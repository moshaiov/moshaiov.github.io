\documentclass{book}
\usepackage{amssymb, amsmath, xcolor, tikz, tikz-cd}
\usetikzlibrary{automata}

\usepackage[margin=1.5cm]{geometry}
\renewcommand{\baselinestretch}{1.5}

\newcommand{\R}{\mathbf{R}}
\newcommand{\Z}{\mathbf{Z}}
\newcommand{\biject}{\longleftrightarrow}
\renewcommand{\to}{\rightarrow}
\newcommand{\from}{\leftarrow}
\newcommand{\eps}{\varepsilon}
\newcommand{\inv}{^{-1}}
\newcommand{\sub}{\subseteq}
\newcommand{\blue}[1]{{\color{blue}{{#1}}}}
\newcommand{\orange}[1]{{\color{orange}{{#1}}}}
\newcommand{\red}[1]{{\color{red}{{#1}}}}
\newcommand{\green}[1]{{\color{green}{{#1}}}}
\newcommand{\dfn}{\orange{definition }}
\newcommand{\cor}{\orange{corollary }}
\newcommand{\exr}{\orange{exercise }}
\newcommand{\theorem}{\orange{theorem }}
\newcommand{\claim}{\orange{claim }}
\newcommand{\exm}{\orange{example }}
\newcommand{\exms}{\orange{examples }}
\newcommand{\obs}{\orange{observation }}
\newcommand{\proof}{\orange{proof }}
\newcommand{\sol}{\orange{solution }}
\newcommand{\nott}{\green{notation }}
\newcommand{\warn}{\red{warning }}
\begin{document}
%COMBINATORICS
\chapter*{combinatorics}
\newpage
%LECTURE 1
\section*{lecture 1 - basic enumeration}
\blue{principle} if two sets $A$ and $B$ are in one to one correspondence, they have the same cardinality. \\
\blue{principle} if $f:A\to B$ is a function such that each $b\in B$ has precisely $k$ preimages, then $|A|=k|B|$. \\
\blue{principle} if $R\sub A\times B$ is a relation such that each $a\in A$ relates to precisely $m$ elements in $B$ and each $b\in B$ relates to precisely $k$ elements in $A$, then $m|A|=k|B|=|R|$. \\
\exm $\{\},\{1\},\{2\},\{3\},\{1,2\},\{1,3\},\{2,3\},\{1,2,3\}$ are the eight subsets of $[3]$. \\
\exr how many subsets of $[n]=\{1,2,3,\dots,n\}$ are there. \\
\sol $2^n$. to specify a set $A\sub[n]$ we must specify for each $i\in[n]$, independently, wether or not it belongs to $A$. in other words, the collection of subsets of $[n]$ is in one to one correspondence with binary sequences of length $n$. for instance $(0,0,1,0,1,1,1,0)$ corresponds to the subset $\{3,5,6,7\}$ of $[8]$. \\
\exm $\{\},\{1,2\},\{1,3\},\{2,3\}$ are the four subsets of $[3]$ of even cardinality, and $\{1\},\{2\},\{3\},\{1,2,3\}$ are the four subsets of odd cardinality. \\
\exr the number of subsets of $[n]$ with an even cardinality is the number of subsets with an cardinality. \\
\sol as above, subsets of $[n]$ correspond to binary sequences of length $n$. the cardinality of a subset corresponds to the number of lit bits in the binary sequence. thus, flipping the first bit yields a one to one correspondence between sequences with an even number of lit bits and those with an odd number of lit bits. so flipping the inclusion of $1$ yields a one to one correspondence between subsets with even cardinality and those with odd cardinality. for instance, for $n=3$ $
\begin{matrix}
\{\} & \{1,2\} & \{1,3\} &\{2,3\}\\
\{1\}&\{2\}&\{3\}&\{1,2,3\}
\end{matrix}$\\
\orange{note} if $n$ is odd, complementing a set $A\biject A^\complement$ yields a one to one correspondence between sets of even and sets of odd cardinality in $[n]$. for instance, for $n=3$\\ $
\begin{matrix}
\{\} & \{1,2\} & \{1,3\} &\{2,3\}\\
\{1,2,3\}& \{3\}&\{2\}&\{1\}
\end{matrix}$\\
\exm the six \blue{permutations} of $[3]$ are $(1,2,3)$, $(1,3,2)$, $(2,1,3)$, $(2,3,1)$, $(3,1,2)$, $(3,2,1)$. \\
\exr how many {permutations} of $[n]$ are there. \\
\sol let $\text{fact}(n)$ be the number of permutations of $[n]$. for a general permutation of $[n]$, the first letter $i\in[n]$ has a total of $n$ choices. for each such choice, the number of permutations starting with $i$ is the number of ways permute the remaining $n-1$ elements $[n]\setminus\{i\}$ in the remaining $n-1$ cells. thus $\text{fact}(n)=n\cdot\text{fact}(n-1)$. since $\text{fact}(1)=1$, $\text{fact}(n)=n!=n(n-1)\dots 1$ is the number of permutations of $[n]$. \\
\exm the six subsets of $[4]$ of cardinality $2$ are $\{1,2\},\{1,3\},\{1,4\},\{2,3\},\{2,4\},\{3,4\}$. \\
\exm $(1,2),(1,3),(2,1),(2,3),(3,1),(3,2)$ are the six sequences without repetitions with elements from $[3]$. \\
\exr how many sequences of length $k$ are there without repetitions and elements from $[n]$. \\
\sol\orange{1} the first element has $n$ choices $i\in[n]$. for each such choice, it remains to pick a sequence of length $k-1$ with elements from $[n]\setminus\{i\}$ to complete it. thus there are $n(n-1)\dots(n-(k-1))$ such sequences. \\
\sol\orange{2} consider corresponding such a sequence to a permutation of $[n]$ if the permutation starts as the sequence. for instance, for $k=3,n=6$, the sequence without repetitions $(5,2,3)$ corresponds to the permutation $(5,2,3,1,4,6)$, among others. since each starting sequence corresponds to precisely $(n-k)!$ permutations, the number of such sequences is $\dfrac{n!}{(n-k)!}=n(n-1)\dots(n-(k-1))$. \\
\newpage
%SHEET 1
\section*{sheet 1}
\orange{problem} how many subsets of $[n]$ of cardinality $k$ are there. \\
\orange{problem} how many strictly increasing sequences of integers $a_1< \dots < a_k$ are there with $1\le a_1, a_k\le n$. \\
\nott ${n\choose k}=\frac{n!}{k!(n-k)!}$\\
\orange{problem} show that $\sum_{k=0}^n {n\choose k}=2^n$ and $\sum_{k=0}^n (-1)^k{n\choose k}=0$. \\
\orange{problem} show more generally that $(x+y)^n=\sum_{k=0}^n {n\choose k}x^ky^{n-k}$. \\
\orange{problem} show that $\#\text{odd cardinality subsets of }[n]=\#\text{even cardinality subsets of }[n]$ also by induction on $n$. \\
\orange{problem} show that given $k\le n$ we have $ 2^{n-k}{n\choose k}=\sum_{m=k}^n {n\choose m}{m\choose k}$ both by counting and algebraically.\\
\orange{problem} show that ${n\choose k}={n-1\choose k}+{n-1\choose k-1}$ both by counting and algebraically. \\
\orange{problem} show that $(k+1){n\choose k+1}=(n-k){n\choose k}$ both by counting and algebraically.\\
\newpage
%LECTURE 2
\section*{lecture 2 - stars and bars, Catalan}
\exm the solutions to $x_1+x_2+x_3=5$ in positive integers are $(1,1,3),(1,2,2),(1,3,1),(2,1,2),(2,2,1),(3,1,1)$. \\
\theorem the number of solutions to $x_1+\dots+x_{k+1}=n+1$ in positive integers $x_j$ is ${n\choose k}$. \\
\proof consider $n+1$ consecutive ones, and $n$ plus signs between them. a selection of $k$ of the $n$ signs is effectively a separation of the $n+1$ ones into $k+1$ parts. for example, $4+2+1+2=9$ corresponds to $1+1+1+1 \blue{+}1+1 \blue{+}1 \blue{+}1+1=9$.\\
\nott suppose there are $n$ plus and minus signs on a circle. call a sign \green{very positive} if the sum of any sequence starting at this sign and going clockwise is positive. \\
\exm there is precisely one {very positive} sign in the configuration 
\begin{tikzpicture}[->,scale=.5]
   \node (i) at (0:1cm)  {$+$};
   \node (j) at (72:1cm) {$-$};
   \node (k) at (144:1cm) {$+$};
   \node (k) at (216:1cm) {$+$};
   \node (k) at (288:1cm) {$-$};
\end{tikzpicture} and three in \begin{tikzpicture}[->,scale=.5]
   \node (i) at (0:1cm)  {$-$};
   \node (j) at (51:1cm) {$+$};
   \node (k) at (103:1cm) {$+$};
   \node (k) at (154:1cm) {$+$};
   \node (k) at (206:1cm) {$-$};
   \node (k) at (257:1cm) {$+$};
   \node (k) at (309:1cm) {$+$};
\end{tikzpicture}
and in \begin{tikzpicture}[->,scale=.5]
   \node (i) at (0:1cm)  {$-$};
   \node (j) at (51:1cm) {$+$};
   \node (k) at (103:1cm) {$+$};
   \node (k) at (154:1cm) {$+$};
   \node (k) at (206:1cm) {$+$};
   \node (k) at (257:1cm) {$+$};
   \node (k) at (309:1cm) {$-$};
\end{tikzpicture}\\
\theorem the number of very positive signs is independent of the order of the signs. it is given by $p-m$ where $p=\#+$ and $m=\#-$. [except if $m>p$, in which case it's zero]. \\
\proof if $p,m>0$, there must be a plus sign followed (in clockwise order) by a minus sign. deleting the two won't change the very positive signs! thus, without changing the number of very positive signs, we may change delete $\pm$ pairs until we remain with $p-m$ (very positive) plusses. [the case $m>p$ means the sum on the whole circle is $\le0$, so no sign is very positive] \\
\dfn a \blue{Catalan} sequence is a sequence $\eps_1,\dots,\eps_{2n}$ consisting of $n$ plus ones and $n$ minus ones such that $0\le \eps_1+\dots+\eps_k$ for all $k\in[2n]$.\\
\cor the number {Catalan} sequences is $\dfrac{{2n+1\choose n}}{2n+1}$. \\
\proof there are $2n+1\choose n$ sequences of length $2n+1$ with $n+1$ plus ones and $n$ minus ones. the theorem implies that for each such sequence $(a_1,\dots,a_{2n+1})$ there is a unique index $i\in[2n+1]$ such that $a_i,a_i+a_{i+1},\dots,a_i+a_{i+1}+\dots+a_{i-1}$ are all positive [indices mod $2n+1$]. that is, there is a unique $i\in[2n+1]$ with $(a_{i+1},\dots,a_{i-1})$ Catalan. mapping $(a_1,\dots,a_{2n+1})\mapsto i,(a_{i+1},\dots,a_{i-1})$ (for said index $i$) is injective, and onto $[2n+1]\times \{\text{Catalan}\}$. thus ${2n+1\choose n }= (2n+1)\#\text{Catalan}$. \\
\newpage
%SHEET 2
\section*{sheet 2}
\orange{problem} find the number of solutions to $x_1+\dots+x_k\le n$ in positive integers. \\
\orange{problem} find the number of subsets of $[n]$ of cardinality $k$ which contain no two consecutive numbers. \\
\orange{problem} each Catalan sequence has a unique representation of the form $(+1,c_1,-1,c_2)$ where $c_1,c_2$ are (possibly empty) Catalan sequences. \\
\orange{problem} $C_n=\sum_{k=0}^{n-1}C_kC_{n-1-k}$ where $C_j=\text{Catalan}_j$. \\
\end{document}
