definitions. 
the differentials $\d z$ and $\d\bar z$ are given by $\d z = \d x+i\d y$ and $\d \bar z=\d x - i \d y$. 
the differential operators $\partial,\bar\partial$ are given by $\partial = \frac{1}{2}(\partial_x - i\partial_y)$ and $\bar\partial=\frac{1}{2}(\partial_x + i\partial_y)$.
\\
\\
observations. \\
$\d f=\partial_x f \d x+\partial_y f \d y = \partial f \d z + \bar\partial f \d\bar z$. \\
$f$ satisfies the Cauchy Riemann equations iff $\bar\partial f= 0$. \\
$\d (f \d z)=2i\bar\partial f \d x\wedge\d y$
\\
\\
corollary. let $D$ be a disk, let $f:\bar D\to \C$ be continuous and holomorphic on $D$. then $\oint_{\partial D} f\d z = 0$.



let $x_{n}+1 = x_{n-1}x_{n+1}$. then $x$ has period five.
let $|x_{n}|=x_{n-1}+x_{n+1}$. then $x$ has period nine.



Klein
notation. \varsigma=x^{11}y+11x^{6}y^{6}-xy^{11}
exercise. the roots of \varsigma are [x:y]=0,\infty,2e^{2\pi ik/5}\cos(2\pi/5),2e^{2\pi ik/5}\cos(4\pi/5). these twelve roots in \widehat{\C} map to the vertices of an icosahedron in \S^{2} (by stereographic projection).
notations. h=\det\left(\begin{array}{cc}
\varsigma_{xx} & \varsigma_{xy}\\
\varsigma_{yx} & \varsigma_{yy}
\end{array}\right) and J=\det\left(\begin{array}{cc}
\varsigma_{x} & \varsigma_{y}\\
h_{x} & h_{y}
\end{array}\right).

